\documentclass[12pt]{article}

\usepackage{amsfonts,bm}
\usepackage{graphics,graphicx}
%\documentclass[pre,showpacs]{revtex4-1}
%\documentclass[pre,twocolumn,showpacs]{revtex4}
%\documentclass[12pt]{article}
%\usepackage[english]{babel}
%\usepackage{amsmath,amssymb,amsfonts}
%\usepackage[dvips]{graphicx}
\usepackage{amsmath}
\usepackage{color}
%\usepackage{fancyhdr}
%\usepackage{showlabels}
\usepackage{tikz}

\newcommand{\D}[1]{\frac{\partial}{\partial{#1}}}
\newcommand{\Dd}[2]{\frac{\partial{#1}}{\partial{#2}}}
\newcommand{\then}{\Rightarrow}
\newcommand{\deriv}[1]{\frac{d}{d{#1}}}
\newcommand{\derivd}[2]{\frac{d{#1}}{d{#2}}}
\newcommand{\derivdd}[2]{\frac{d^2{#1}}{d{#2}^2}}


%%%%%%%%%%tikz                                                                  
\newcommand{\ancho}{1.2}                                                        
\newcommand{\rad}{1.4}                                                          
\newcommand{\dbit}{0.56}

\tikzstyle{rect}=[text width=\ancho cm,                                         
  rectangle,fill=blue!70!red!10!white,align=left,text=black,font=\bfseries]    
\tikzstyle{white}=[fill=white,align=center,text depth=0.3ex,rectangle,
  text=black,font=\bfseries] 


\begin{document}

\title{Using a soft circle to represent the location of the pendulum mass}
\author{Carlos E. \'Alvarez}
%\date{}
\maketitle

\section{Soft circle function}
The soft circle function centered at $c$ is defined as the product of two sigmoid functions:
\begin{align}
  f(x)&=\left(\frac{1}{1+e^{-A((x-c)+R)}}\right)\left(\frac{1}{1+e^{A((x-c)-R)}}\right)\nonumber\\
  &=\frac{1}{1+e^{-2AR}+2e^{-AR}\cosh(A(x-c))},
  \label{softc}
\end{align}
where $A$ controls the decay of the rim of the circle, and $R$ the radius of the circle.\\

\section{Soft circle as pendulum mass}
In 2D, to represent the pendulum mass as a soft circle we realize that the center of the pendulum is located at
\begin{align}
  \vec{r}=(r_x,r_y)=(p_x+l\sin\theta,p_y+l\cos\theta),
\end{align}
where $(p_x,p_y)$ is the location of the pivot of the pendulum, $l$ is the length of the pendulum, and $\theta$ is the angle the pendulum makes with the vertical. Which means we have to compute function (\ref{softc}) as a function of the radial distance $d$ from $\vec{r}$, instead of $c$. That is
\begin{align}
  D = \sqrt{(x-p_x-l\sin\theta)^2+(y-p_y-l\cos\theta)^2}.
\end{align}
Then
\begin{align}
  f(D) = \frac{1}{a+b\cosh(AD)},
\end{align}
where
\begin{align}
  a &= 1+e^{-2AR}\\
  b &= 2e^{-AR}.
\end{align}

The angle $\theta$ is a function of time, and we want to compute the first and second time derivatives of $f$ with respect to time.

\subsection{1st derivative}
\begin{align}
  \frac{df}{dt}=\frac{df}{dD}\frac{dD}{d\theta}\dot{\theta}.
\end{align}

\begin{align}
  \frac{df}{dD}=-\frac{bA\sinh(AD)}{\left[a+b\cosh(AD)\right]^2}=-bA\sinh(AD)f(D)^2.
\end{align}

\begin{align}
  \frac{dD}{d\theta}=-\frac{l}{D}[(y-p_y)\sin\theta-(x-p_x)\cos\theta].
\end{align}

Then
\begin{align}
  \dot{f}(D)=\frac{df}{dt}=\frac{lbA}{D}f(D)^2\sinh(AD)[(x-p_x)\cos\theta-(y-p_y)\sin\theta]\dot{\theta}
\end{align}

\subsection{2nd derivative}
\begin{align}
  \frac{d^2f}{dt^2}=&lbA\frac{d}{dt}\left[\frac{1}{D}\right]f(D)^2\sinh(AD)[(y-p_y)\sin\theta-(x-p_x)\cos\theta)]\dot{\theta}\nonumber\\
  &+lbA\frac{1}{D}\frac{d}{dt}\left[f(D)^2\right]\sinh(AD)[(y-p_y)\sin\theta-(x-p_x)\cos\theta)]\dot{\theta}\nonumber\\
  &+lbA\frac{1}{D}f(D)^2\frac{d}{dt}\left[\sinh(AD)\right][(y-p_y)\sin\theta-(x-p_x)\cos\theta)]\dot{\theta}\nonumber\\
  &+lbA\frac{1}{D}f(D)^2\sinh(AD)\frac{d}{dt}\left[(y-p_y)\sin\theta-(x-p_x)\cos\theta)\right]\dot{\theta}\nonumber\\
  &+lbA\frac{1}{D}f(D)^2\sinh(AD)\left[(y-p_y)\sin\theta-(x-p_x)\cos\theta)\right]\ddot{\theta}\nonumber\\
  =&\frac{l^2bA}{D^3}f(D)^2\sinh(AD)[(y-p_y)\sin\theta-(x-p_x)\cos\theta]^2\dot{\theta}^2\nonumber\\
  &+2\frac{lbA}{D}f(D)\dot{f}(D)\sinh(AD)[(y-p_y)\sin\theta-(x-p_x)\cos\theta]\dot{\theta}\nonumber\\
  &-\frac{lbA^2}{D}f(D)^2\cosh(AD)[(y-p_y)\cos\theta+(x-p_x)\sin\theta][(y-p_y)\sin\theta-(x-p_x)\cos\theta]\dot{\theta}^2\nonumber\\
  &-\frac{lbA}{D}f(D)^2\sinh(AD)\left[(y-p_y)\cos\theta+(x-p_x)\sin\theta\right]\dot{\theta}^2\nonumber\\
  &+\frac{lbA}{D}f(D)^2\sinh(AD)\left[(y-p_y)\sin\theta-(x-p_x)\cos\theta\right]\ddot{\theta}\nonumber\\
  =&\dot{f}(D)\left[\frac{l}{D^2}\dot{\theta}+2\frac{\dot{f}(D)}{f(D)}-\frac{\sin\theta}{\dot{\theta}}\right.\nonumber\\
    &\left.-\left(\frac{A}{\tanh(AD)}+\frac{1}{[(y-p_y)\sin\theta-(x-p_x)\cos\theta)]}\right)\left[(y-p_y)\cos\theta+(x-p_x)\sin\theta\right]\dot{\theta}\right],
\end{align}
where we have used that, for a pendulum with $l=g$
\begin{align}
  \ddot{\theta}=-\sin\theta.
\end{align}

Numerically, it is better to avoid dividing by $\dot{\theta}$ or $f(D)$, so
\begin{align}
  \ddot{f}=&\frac{lbA}{D}f(D)\sinh(AD)[(x-p_x)\cos\theta-(y-p_y)\sin\theta]\nonumber\\
  &\times\left[\frac{l}{D^2}f(D)\dot{\theta}^2+2\dot{f}(D)-f(D)\sin\theta\right.\nonumber\\
  &\left.-\left(\frac{A}{\tanh(AD)}+\frac{1}{[(y-p_y)\sin\theta-(x-p_x)\cos\theta)]}\right)\left[(y-p_y)\cos\theta+(x-p_x)\sin\theta\right]f(D)\dot{\theta}^2\right]\right]
\end{align}

\end{document}
